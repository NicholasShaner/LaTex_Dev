% ----------------------------------------------------------------
% AMS-LaTeX Paper ************************************************
% **** -----------------------------------------------------------
\documentclass{amsart}
%\usepackage{txfonts}
\usepackage{graphicx}
\usepackage{enumitem}
\usepackage{amsmath}
\usepackage{amssymb}
\usepackage{setspace}
% ----------------------------------------------------------------
\vfuzz2pt % Don't report over-full v-boxes if over-edge is small
\hfuzz2pt % Don't report over-full h-boxes if over-edge is small
% THEOREMS -------------------------------------------------------
\newtheorem{thm}{Theorem}[section]
\newtheorem{cor}[thm]{Corollary}
\newtheorem{lem}[thm]{Lemma}
\newtheorem{prop}[thm]{Proposition}
\theoremstyle{definition}
\newtheorem{defn}[thm]{Definition}
\theoremstyle{Exercise}
\newtheorem{ex}[thm]{Exercise}
\theoremstyle{remark}
\newtheorem{rem}[thm]{Remark}
\theoremstyle{rule}
\newtheorem{rul}[thm]{Rule}

\numberwithin{equation}{section}
% MATH -----------------------------------------------------------
\newcommand{\norm}[1]{\left\Vert#1\right\Vert}
\newcommand{\abs}[1]{\left\vert#1\right\vert}
\newcommand{\set}[1]{\left\{#1\right\}}
\newcommand{\Real}{\mathbb R}
\newcommand{\Z}{\mathbb Z}
\newcommand{\To}{\longrightarrow}
\newcommand{\BX}{\bB(X)}
\newcommand{\A}{\mathcal{A}}
% ----------------------------------------------------------------

% define some simple, commonly-used commands
\newcommand{\eps}{\varepsilon}
\newcommand{\dsum}{\displaystyle\sum}
\newcommand{\dint}{\displaystyle\int}

\newcommand{\pdr}[2]{\dfrac{\partial{#1}}{\partial{#2}}}
\newcommand{\pdrr}[2]{\dfrac{\partial^2{#1}}{\partial{#2}^2}}
\newcommand{\pdrt}[3]{\dfrac{\partial^2{#1}}{\partial{#2}{\partial{#3}}}}
\newcommand{\dr}[2]{\dfrac{d{#1}}{d{#2}}}
\newcommand{\aver}[1]{\langle {#1} \rangle}
\newcommand{\Baver}[1]{\Big\langle {#1} \Big\rangle}

\newcommand{\bzero}{\mathbf 0}
\newcommand{\bGamma}{\mbox{\boldmath{$\Gamma$}}}
\newcommand{\btheta}{\boldsymbol \theta}
\newcommand{\bchi}{\mbox{\boldmath{$\chi$}}}
\newcommand{\bnu}{\boldsymbol \nu}
\newcommand{\bmu}{\boldsymbol \mu}
\newcommand{\brho}{\mbox{\boldmath{$\rho$}}}
\newcommand{\bxi}{\boldsymbol \xi}
\newcommand{\bnabla}{\boldsymbol \nabla}
\newcommand{\bOm}{\boldsymbol \Omega}
\newcommand{\blambda}{\boldsymbol \lambda}
\newcommand{\bsigma}{\boldsymbol \sigma}

\newcommand{\bbR}{\mathbb R}
\newcommand{\bbC}{\mathbb C}
\newcommand{\bbQ}{\mathbb Q}
\newcommand{\bbN}{\mathbb N}
\newcommand{\bbZ}{\mathbb Z}

\newcommand{\ba}{\mathbf a} \newcommand{\bb}{\mathbf b}
\newcommand{\bc}{\mathbf c} \newcommand{\bd}{\mathbf d}
\newcommand{\be}{\mathbf e} \newcommand{\bff}{\mathbf f}
\newcommand{\bg}{\mathbf g} \newcommand{\bh}{\mathbf h}
\newcommand{\bi}{\mathbf i} \newcommand{\bj}{\mathbf j}
\newcommand{\bk}{\mathbf k} \newcommand{\bl}{\mathbf l}
\newcommand{\bm}{\mathbf m} \newcommand{\bn}{\mathbf n}
\newcommand{\bo}{\mathbf o} \newcommand{\bp}{\mathbf p}
\newcommand{\bq}{\mathbf q} \newcommand{\br}{\mathbf r}
\newcommand{\bs}{\mathbf s} \newcommand{\bt}{\mathbf t}
\newcommand{\bu}{\mathbf u} \newcommand{\bv}{\mathbf v}
\newcommand{\bw}{\mathbf w} \newcommand{\bx}{\mathbf x}
\newcommand{\by}{\mathbf y} \newcommand{\bz}{\mathbf z}
\newcommand{\bA}{\mathbf A} \newcommand{\bB}{\mathbf B}
\newcommand{\bC}{\mathbf C} \newcommand{\bD}{\mathbf D}
\newcommand{\bE}{\mathbf E} \newcommand{\bF}{\mathbf F}
\newcommand{\bG}{\mathbf G} \newcommand{\bH}{\mathbf H}
\newcommand{\bI}{\mathbf I} \newcommand{\bJ}{\mathbf J}
\newcommand{\bK}{\mathbf K} \newcommand{\bL}{\mathbf L}
\newcommand{\bM}{\mathbf M} \newcommand{\bN}{\mathbf N}
\newcommand{\bO}{\mathbf O} \newcommand{\bP}{\mathbf P}
\newcommand{\bQ}{\mathbf Q} \newcommand{\bR}{\mathbf R}
\newcommand{\bS}{\mathbf S} \newcommand{\bT}{\mathbf T}
\newcommand{\bU}{\mathbf U} \newcommand{\bV}{\mathbf V}
\newcommand{\bW}{\mathbf W} \newcommand{\bX}{\mathbf X}
\newcommand{\bY}{\mathbf Y} \newcommand{\bZ}{\mathbf Z}

\newcommand{\cA}{\mathcal A} \newcommand{\cB}{\mathcal B}
\newcommand{\cC}{\mathcal C} \newcommand{\cD}{\mathcal D}
\newcommand{\cE}{\mathcal E} \newcommand{\cF}{\mathcal F}
\newcommand{\cG}{\mathcal G} \newcommand{\cH}{\mathcal H}
\newcommand{\cI}{\mathcal I} \newcommand{\cJ}{\mathcal J}
\newcommand{\cK}{\mathcal K} \newcommand{\cL}{\mathcal L}
\newcommand{\cM}{\mathcal M} \newcommand{\cN}{\mathcal N}
\newcommand{\cO}{\mathcal O} \newcommand{\cP}{\mathcal P}
\newcommand{\cQ}{\mathcal Q} \newcommand{\cR}{\mathcal R}
\newcommand{\cS}{\mathcal S} \newcommand{\cT}{\mathcal T}
\newcommand{\cU}{\mathcal U} \newcommand{\cV}{\mathcal V}
\newcommand{\cW}{\mathcal W} \newcommand{\cX}{\mathcal X}
\newcommand{\cY}{\mathcal Y} \newcommand{\cZ}{\mathcal Z}

%%%%%%%%%%%%%%Start%%%%%%%%%%%%%Start%%%%%%%%%%%Start%%%%%%%%%%%%%%%Start%%%%%%%%%%%%%%%%%%%%%%%%%Start%%%%%%%%%%%%%%%%
%%%%%%%%%%%%%%Start%%%%%%%%%%%%%Start%%%%%%%%%%%Start%%%%%%%%%%%%%%%Start%%%%%%%%%%%%%%%%%%%%%%%%%Start%%%%%%%%%%%%%%%%
%%%%%%%%%%%%%%Start%%%%%%%%%%%%%Start%%%%%%%%%%%Start%%%%%%%%%%%%%%%Start%%%%%%%%%%%%%%%%%%%%%%%%%Start%%%%%%%%%%%%%%%%
\usepackage{fancyhdr}

\pagestyle{fancy}
\fancyhf{}
\rhead{}
\chead{\includegraphics[scale=.1]{snhu_logo.png}}


\begin{document}
\title{\sf Module Two Problem Set}

\begin{center}
\includegraphics[scale=.1]{snhu_logo.png}
\end{center}

%\thm{bbjh}
\maketitle
This document is proprietary to Southern New Hampshire University. It and the problems within may not be posted on any non-SNHU website.\\\\\\\\
\begin{center}\doublespacing
%Enter your name below this line:
2-3 Problem Set\\
Nicholas Shaner\\
SNHU\\
MAT-230-15848-M01 Discrete Mathematics\\
Professor Kirsten Messer\\
11/10/2024\\
\end{center}


\begin{center}
\rule{\textwidth}{0.4pt}
\end{center}


\newpage


\newpage

\section*{}
\section*{}

Directions: Type your solutions into this document and be sure to show all steps for arriving at your solution. Just giving a final number may not receive full credit.\\

\section*{Problem 1}
\subsection*{Part 1}
{\bf Indicate whether the argument is valid or invalid. For valid arguments, prove that the argument is valid using a truth table. For invalid arguments, give truth values for the variables showing that the argument is not valid.}\\
 \begin{enumerate}

\item \[
\begin{array}{||c||}
\hline \hline
(p \land q) \to r\\
\\
\therefore (p \lor q) \to r\\
\hline \hline
\end{array}
\]\\\\
 %Enter your answer below this comment line.

\begin{displaymath}
  \begin{array}{|c c c | c | c | c | c |}
    p & q & r & p \land q & p \lor q & p \land q \to r & p \lor q \to r\\
    \hline
    T & T & T & T & T & T & T\\
    T & T & F & T & T & F & F\\
    T & F & T & F & T & T & T\\
    T & F & F & F & T & T & F\\
    F & T & T & F & T & T & T\\
    F & T & F & F & T & T & T\\
    F & F & T & F & F & T & T\\
    F & F & F & F & F & T & T\\
  \end{array}
\end{displaymath}
  \\\\
  - In review of the truth table, the only rows that suffices the hypthesis $ (p \land q)$ are 1 and 2, however row 2 proves the argument is invalid due to r = F and $ (p \land q) = T$\\
  \\\\
    \end{enumerate}

\subsection*{Part 2}
{\bf Converse and inverse errors are typical forms of invalid arguments. Prove that each argument is invalid by giving truth values for the variables showing that the argument is invalid. You may find it easier to find the truth values by constructing a truth table.}\\
 \begin{enumerate}[label=(\alph*)]
\item Converse error
\[
\begin{array}{||c||}
\hline \hline
p \to q\\
q\\
\\
\therefore p\\
\hline \hline
\end{array}
\]\\\\
%Enter your answer below this comment line.
\begin{displaymath}
  \begin{array}{| c c | c |}
    p & q & p \to q\\
    \hline
    T & T & T\\
    T & F & F\\
    F & T & T\\
    F & F & T\\
  \end{array}
\end{displaymath}
 \\\\
 - In review of the truth table, both hypotheses are true in rows 1 and 3, however, the conclusion in row 3 is false indicating that the argumant is invalid.
 \\\\
\item Inverse error
\[
\begin{array}{||c||}
\hline \hline
p \to q\\
\neg p\\
\\
\therefore \neg q\\
\hline \hline
\end{array}
\]\\\\
%Enter your answer below this comment line.
\begin{displaymath}
  \begin{array}{| c | c | c | c | c |}\\
    p & q & p \to q & \neg p & \neg q\\
    \hline
    T & T & T & F & F\\
    T & F & F & F & T\\
    F & T & T & T & F\\
    F & F & T & T & T\\
  \end{array}
\end{displaymath}
\\\\
- In review of the truth table, row 3 indicated that when p = F, and q = T, the hypotheses $ p \to q $ and $\neg p $ are both True, however the conclusion $\neg q $ is False, this indicates that the argument is invalid. 
\\\\
\end{enumerate}

\subsection*{Part 3}
{\bf Which of the following arguments are invalid and which are valid? Prove your answer by replacing each proposition with a variable to obtain the form of the argument. Then prove that the form is valid or invalid.}\\
 \begin{enumerate}[label=(\alph*)]
  \item \[
\begin{array}{||c||}
\hline \hline
\text{The patient has high blood pressure or diabetes or both.}\\
\text {The patient has diabetes or high cholesterol or both.}\\
\\
\therefore \text {The patient has high blood pressure or high cholesterol.
}\\
\hline \hline
\end{array}
\]
\vspace*{1in}\\
%Enter your answer below this comment line.
b: The patient has high blood pressure\\
c: The patient has high cholesterol\\
d: The patient has diabetes\\
\\\\
\[
\begin{array}{||c||}
  \hline \hline
  b \lor d\\
  d \lor c\\
  \\
  \therefore b \lor c\\
  \hline \hline
\end{array}
\]\\\\
\begin{displaymath}
  \begin{array}{| c c c c c | c |}
    b & c & d & b\lor d & d \lor c & b \lor c\\
    \hline
    T & T & T & T & T & T\\
    T & T & F & T & T & T\\
    T & F & T & T & T & T\\
    T & F & F & T & F & T\\
    F & T & T & T & T & T\\
    F & T & F & F & T & T\\
    F & F & T & T & T & F\\
    F & F & F & F & F & F\\
  \end{array}
\end{displaymath}
\\\\
- In review of the truth table depicted, is is clearly shown that in rows 4 and 6 that in each conclusion is true but in each respective scenario all the hypotheses are not true.
  In the first scenaio, b = T, c = F, and d = F. This in turn means $b \lor d $ is true, however $d \lor c $ is false.\\In the second scenario, b = F, c = T, and d = F. In this, 
  $b \lor d $ is false, and $d \lor c $ is true.Therefore, the argument is invalid.\\
    \end{enumerate}
 \newpage
%--------------------------------------------------------------------------------------------------

\vspace*{0.5in}
\section*{Problem 2}
\subsection*{Part 1}

 Which of the following arguments are valid? Explain your reasoning.\\
 \begin{enumerate}[label=(\alph*)]
\item I have a student in my class who is getting an $A$. Therefore, John, a student in my class, is getting an $A$. \\\\
%Enter your answer below this comment line.
- It is quickly noticable that this argument is missing critical information. Though it is stated a student in class is getting an $A$, there is no information regarding the other 
studends in the class. Without this missing information, there is no definitive differentiation to imply that John is getting the $A$.
\\\\
\item Every Girl Scout who sells at least 30 boxes of cookies will get a prize. Suzy, a Girl Scout, got a prize. Therefore, Suzy sold at least 30 boxes of cookies.\\\\
%Enter your answer below this comment line.
- This argument is invalid. Logically, the argument consists of $ \forall x (P(x) \rightarrow Q(x))$ where if variable x sold 30 boxes of cookies, then variable x received a prize.
In evaluating the the scenario (P(Suzy) $\to$ S(Suzy)), there would be a potential situation where Suzy sold 30 boxes of cookies but did not get a prize. This would indicate though the
hypothesis is true, the conclusion is false, meaning the argument is invalid.
\\\\
 \end{enumerate}

 \subsection*{Part 2}
Determine whether each argument is valid. If the argument is valid, give a proof using the laws of logic. If the argument is invalid, give values for the predicates $P$ and $Q$ over 
the domain ${a,\; b}$ that demonstrate the argument is invalid.\\
 \begin{enumerate}[label=(\alph*)]
\item \[
\begin{array}{||c||}
\hline \hline
\exists x\, (P(x)\; \land \;Q(x) )\\
\\
\therefore \exists x\, Q(x)\; \land\; \exists x \,P(x) \\
\hline \hline
\end{array}
\]\\\\
 %Enter your answer here.
 - This argument is valid. In english, the argument reads "There exists an instance of x in the domain that satisfies P(x) and Q(x)". By applying the commutative law:\\ $P(x) \land Q(x) \equiv Q(x) \land P(x)$\\\\
 \vspace*{0.5in}\\
 - Then applying the distributative law:\\ $\exists x (Q(x) \land P(x)) \equiv \exists xQ(x) \land \exists xP(x)$, which reads "There exists an instance of x in the domain that satifies P(x) and an existance of the same instance x in the domain that satisfies Q(x)"
 \\\\


\item \[
\begin{array}{||c||}
\hline \hline
\forall x\, (P(x)\; \lor \;Q(x) )\\
\\
\therefore \forall x\, Q(x)\; \lor \; \forall x\, P(x) \\
\hline \hline
\end{array}
\]\\\\
 %Enter your answer here.
 - This argument is invalid. The hypothesis $\forall x (P(x) \lor Q(x))$, reads that for every element x in the domain either P(x) OR Q(x) is true. The conclusion $\forall x Q(x) \lor \forall x P(x)$, reads that every 
 element in the domain equates $Q(x)$ to True or every element in the domain equates $P(x)$ to True. Though every element x in the domain may satisfy either P(x) or Q(x) does not mean that 
 every element x in the domain will satisfy Q(x) or every element in the domain will satisfy P(x)\\
 Graphically,
 \[
 \begin{array}{|c | c c|}
   & Q(x) & P(x)\\
  \hline
  a & T & F\\
  b & F & T\\
 \end{array}
 \]
 This shows that within the domain \{a, b\} that not every element satisfies $Q(x)$ as $Q(b)$ is false, and not every element satisfies $P(x)$ as $P(a)$ is false. Therefore, the argument is invalid.
 \\\\
 \end{enumerate}
 \newpage
%--------------------------------------------------------------------------------------------------

\vspace*{0.5in}
\section*{Problem 3}

Prove the following using a direct proof. Your proof should be expressed in complete English sentences.
\\\\

If $a$, $b$, and $c$ are integers such that $b$ is a multiple of $a^3$ and $c$ is a multiple of $b^2$, then $c$ is a multiple of $a^6$.
\\\\
%Enter your answer below this comment line.
\textbf{Assumptions:}\\
There exists an integer m, such that $b = m * a^3$\\
There exists an integer n, such that $c = n * b^2$\\
\textbf{Proof:}\\
- Let a, b, and c be integers, and let b be a multiple of $a^3$, and c be a multiple of $b^2$. We will prove that c is a multiple of $a^6$.\\
- By substituting b into the expression $c = n * b^2$, we get $n * (m * a^3)^2$. Simplified, $c = n * (m^2 * a^6)$\\
- Since n and m are integers $n*m^2$ is also an integer\\
- Therefore, $c$ is equal to an integer multiple of $a^6$ \qedsymbol\\
\\\\
 \newpage
%--------------------------------------------------------------------------------------------------
\vspace*{0.5in}
\section*{Problem 4}
Prove the following using a direct proof:
\\

The sum of the squares of 4 consecutive integers is an even integer.
%Enter your answer below this comment line.
\\\\
\textbf{Assumption:}\\
There exists an integer x.\\
$x, x_2, x_3, x_4$ represent four consecutive integers.\\
\textbf{Proof:}\\
- Let $x_2 = (x+1), x_3 = (x+2),$ and $x_4 = (x+3)$. We will prove that the sum of the squares of four consecutive integers is even.\\
- By evaluating the assumptions $x^2 + (x+1)^2 + (x+2)^2 + (x+3)^2$, we get $4x^2 + 12x + 14$\\
- Factoring the equation: we get $2(2x^2 + 6x + 7)$\\
- Let n be an integer.
- Since x is an integer, $2x^2 + 6x + 7$ is also an integer, such that it can be represented by the integer n.\\
- The equation can be represented by $2n$\\
- Therefore, though integer n can represent an odd or even integer, multiplying by 2 will always evaluate to a an even integer. \qedsymbol
\\\\


 \newpage
%--------------------------------------------------------------------------------------------------
\vspace*{0.5in}
\section*{Problem 5}

Prove the following using a proof by contrapositive:
\\\\

Let $x$ be a rational number. Prove that if $xy$ is irrational, then y is irrational.\\\\
%Enter your answer below this comment line.
\\\\
\textbf{Assumption:}\\
- $y$ is rational.\\
%- $xy$ is irrational.\\
\textbf{Proof:}\\
- Let integers $a,\;b\;,\;c,\;d \in Z$.\\
- Let $x$ be a rational number where as $x = \frac{a}{b}, b \neq 0$, and $y$ is ratioal where as $y = \frac{c}{d}, d \neq 0$. We will use contropositive evaluation to prove that $xy$ is irrational.\\
- Substituting the rational values of $x$ and $y$ into $xy$, we get $xy = \frac{a}{b} * \frac{c}{d} \equiv \frac{ac}{bd}$.\\
- As $a,\;b\;,\;c,\;and\; d$ are all integers in domain $Z$, the numerator $ac$ and denominator $bd$ are also integers in domain $Z$. Which means the $xy$ is also an integer in the domain $Z$.\\
- As stated, $x$ and $y$ are both rational, such that $\{b,d\} \neq 0$, which means $\frac{ac}{bd}$, and as associated $xy$ is also a rational number.\\
- Therefore, $xy$ and $y$ being rational proves the contrapositive of original statement, meaning that if $xy$ is irrational then $y$ is irrational. \qedsymbol
\\\\




 \newpage
%--------------------------------------------------------------------------------------------------
\vspace*{0.5in}
\section*{Problem 6}
Prove the following using a proof by contradiction:
\\\\


The average of four real numbers is greater than or equal to at least one of the numbers.
%Enter your answer below this comment line.
\\\\
\textbf{Proof:}\\
- Using proof of contradiction, we assume the opposite of what is asked. Therefore, "The average of four real numbers is less than each of the numbers."
\\\\
Logically put,\\
$(a + b + c + d)/ 4 < a$\\
$(a + b + c + d)/ 4 < b$\\
$(a + b + c + d)/ 4 < c$\\
$(a + b + c + d)/ 4 < d$
\\\\
This implies:\\
$(a + b + c + d) < 4a$\\
$(a + b + c + d) < 4b$\\
$(a + b + c + d) < 4c$\\
$(a + b + c + d) < 4d$
\\\\
This implication can be assessed as the sum of the four numbers is less than 4 times each number. By combining these statements, we can get:
\\\\
$4(a + b + c + d) < (4a + 4b + 4c + 4c) \equiv 4(a + b + c + d) < 4(a + b + c + d)$
\\\\
Since a number can \textbf{NOT} be less than itself, this means that the assumption "The average of four real numbers is less than each of the numbers" is a contradiction and is false.\\
Therefore, the average of four real numbers must be greater than or equal to at least one of the numbers. \qedsymbol
\\\\



 \newpage
%--------------------------------------------------------------------------------------------------
\vspace*{0.5in}
\section*{Problem 7}

Let $\displaystyle q = \frac{a}{b}$ and $\displaystyle r = \frac{c}{d}$ be two rational numbers written in lowest terms. Let $s = q + r$ and $\displaystyle s = \frac{e}{f}$ be written in lowest terms. Assume that $s$ is not $0$.\\

 Prove or disprove the following two statements.
\\\\
a.  If $b$ and $d$ are odd, then $f$ is odd.
\\\\
b. If $b$ and $d$ are even, then $f$ is even.
\\\\
%Enter your answer below this comment line.
\\\\
\textbf{Proof:}\\
Without loss of generality, we can substitute the values of $q$ and $r$ into $s = q + r$\\
This would give us: $s = q + r = \frac{a}{b} + \frac{c}{d}$\\
As both $q$ and $r$ were stated to be rational numbers, $b\; and\; d\; \neq\; 0$.\\
We can create a common denominator by multiplying the the 2 elements appropriately: $s = (\frac{d}{d})\frac{a}{b} + \frac{c}{d}(\frac{b}{b}) = \frac{ad + cb}{bd}$\\
Assuming $b$ and $d$ are odd, $bd$ is also odd.\\
As stated $s = q + r$ and $s = \frac{e}{f}$, which mean $(bd) = f$, therefore, since $bd$ is odd, $f$ is odd. \qedsymbol
\\\\


\newpage
\vspace*{0.5in}
\section*{Problem 8}
{\bf Define $P(n)$ to be the assertion that:}\\
\[\displaystyle \sum_{j=1}^{n}\, j^2 \;=\;\frac{n(n+1)(2n+1)}{6}\]\\\\
\begin{enumerate}[label=(\alph*)]
  \item Verify that $P(3)$ is true.\\\\
   %Enter your answer here.
   \\\\
   - $P(3) = \sum_{j=1}^{n}\, 3^2 \;=\;\frac{3(3+1)(2(3)+1)}{6} =\; \frac{3(4)(7)}{6} =\; \frac{84}{6} =\; 14$
   \\\\
   - Evaluate the sum of the first 3 squares of $j^2$ for domain of j $\{1,3\}$:\\
   = $1^2+2^2+3^3 = 1+4+9 = \bf{14}$\\\\
   - Therefore, $14\;=\;14$ and $P(3)$ is True.
   \\\\
  \item Express $P(k)$.\\\\
   %Enter your answer here.
   $\sum_{j=1}^{k}\, j^2 \;=\;\frac{k(k+1)(2k+1)}{6}\;=\;\frac{2k^3 + 3k^2 + k}{6}$
   \\\\
  \item Express $P(k+1)$.\\\\
   %Enter your answer here.
   $\sum_{j=1}^{k+1}\, j^2 \;=\;\frac{k+1\big{(}(k+1)+1\big{)(}2(k+1)+1\big{)}}{6}\;=\;\frac{2k^3 + 9k^2 + 13K + 9}{6}$
   \\\\
   \item In an inductive proof that for every positive integer $n$,
   \[\displaystyle \sum_{j=1}^{n}\, j^2 \;=\;\frac{n(n+1)(2n+1)}{6}\]
   what must be proven in the base case?\\\\
    %Enter your answer here.
    The base case is the evaluation of $n=1$:\;$\frac{1(1+1)(2(1)+1)}{6}\;=\;\frac{6}{6}\;=\;1$ 
    \\\\
    \item In an inductive proof that for every positive integer $n$,
   \[\displaystyle \sum_{j=1}^{n}\, j^2 \;=\;\frac{n(n+1)(2n+1)}{6}\]
   \\\\
   what must be proven in the inductive step?\\\\
   %Enter your answer here.
   Assuming P(n) is true, we must prove P(n+1) is also true.
   \\\\
   \item What would be the inductive hypothesis in the inductive step from your previous answer?\\\\
    %Enter your answer here.
    P(n) is true for all positive integers n.
    \\\\
   \item Prove by induction that for any positive integer n,
   \[\displaystyle \sum_{j=1}^{n}\, j^2 \;=\;\frac{n(n+1)(2n+1)}{6}\] \\\\
    %Enter your answer here.
    - As already proven, P(n), where n = 1, evaluates to 1 which validates $j^2$ = $1^2$ = 1.\\
    - P(n+1) = P(2) = $\frac{n+1((n+1)+1)(2(n+1)+1)}{6}\;=\;\frac{2((2)+1)(2(2)+1)}{6}\;=\;\frac{2*3*5}{6}\;=\;\frac{30}{6}\;=\;5$\\
    - The sum of the squares of P(2) = $1^2 + 2^2\;=\;1 + 4\;=\;5$\\
    - Therefore, the evaluation proves that P(n) implies P(n+1).
    \\\\
\end{enumerate}

\end{document}
