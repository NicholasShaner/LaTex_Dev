% ----------------------------------------------------------------
% AMS-LaTeX Paper ************************************************
% **** -----------------------------------------------------------
%\documentclass{amsart}
%\usepackage{txfonts}
%\documentclass[12pt,oneside]{article}
\documentclass{amsart}
\usepackage{graphicx}
\usepackage{enumitem}
\usepackage{setspace}
% ----------------------------------------------------------------
\vfuzz2pt % Don't report over-full v-boxes if over-edge is small
\hfuzz2pt % Don't report over-full h-boxes if over-edge is small
% THEOREMS -------------------------------------------------------
\newtheorem{thm}{Theorem}[section]
\newtheorem{cor}[thm]{Corollary}
\newtheorem{lem}[thm]{Lemma}
\newtheorem{prop}[thm]{Proposition}
\theoremstyle{definition}
\newtheorem{defn}[thm]{Definition}
\theoremstyle{Exercise}
\newtheorem{ex}[thm]{Exercise}
\theoremstyle{remark}
\newtheorem{rem}[thm]{Remark}
\theoremstyle{rule}
\newtheorem{rul}[thm]{Rule}

\numberwithin{equation}{section}
% MATH -----------------------------------------------------------
\newcommand{\norm}[1]{\left\Vert#1\right\Vert}
\newcommand{\abs}[1]{\left\vert#1\right\vert}
\newcommand{\set}[1]{\left\{#1\right\}}
\newcommand{\Real}{\mathbb R}
\newcommand{\Z}{\mathbb Z}
\newcommand{\To}{\longrightarrow}
\newcommand{\BX}{\bB(X)}
\newcommand{\A}{\mathcal{A}}
% ----------------------------------------------------------------

% define some simple, commonly-used commands
\newcommand{\eps}{\varepsilon}
\newcommand{\dsum}{\displaystyle\sum}
\newcommand{\dint}{\displaystyle\int}

\newcommand{\pdr}[2]{\dfrac{\partial{#1}}{\partial{#2}}}
\newcommand{\pdrr}[2]{\dfrac{\partial^2{#1}}{\partial{#2}^2}}
\newcommand{\pdrt}[3]{\dfrac{\partial^2{#1}}{\partial{#2}{\partial{#3}}}}
\newcommand{\dr}[2]{\dfrac{d{#1}}{d{#2}}}
\newcommand{\aver}[1]{\langle {#1} \rangle}
\newcommand{\Baver}[1]{\Big\langle {#1} \Big\rangle}

\newcommand{\bzero}{\mathbf 0}
\newcommand{\bGamma}{\mbox{\boldmath{$\Gamma$}}}
\newcommand{\btheta}{\boldsymbol \theta}
\newcommand{\bchi}{\mbox{\boldmath{$\chi$}}}
\newcommand{\bnu}{\boldsymbol \nu}
\newcommand{\bmu}{\boldsymbol \mu}
\newcommand{\brho}{\mbox{\boldmath{$\rho$}}}
\newcommand{\bxi}{\boldsymbol \xi}
\newcommand{\bnabla}{\boldsymbol \nabla}
\newcommand{\bOm}{\boldsymbol \Omega}
\newcommand{\blambda}{\boldsymbol \lambda}
\newcommand{\bsigma}{\boldsymbol \sigma}

\newcommand{\bbR}{\mathbb R}
\newcommand{\bbC}{\mathbb C}
\newcommand{\bbQ}{\mathbb Q}
\newcommand{\bbN}{\mathbb N}
\newcommand{\bbZ}{\mathbb Z}

\newcommand{\ba}{\mathbf a} \newcommand{\bb}{\mathbf b}
\newcommand{\bc}{\mathbf c} \newcommand{\bd}{\mathbf d}
\newcommand{\be}{\mathbf e} \newcommand{\bff}{\mathbf f}
\newcommand{\bg}{\mathbf g} \newcommand{\bh}{\mathbf h}
\newcommand{\bi}{\mathbf i} \newcommand{\bj}{\mathbf j}
\newcommand{\bk}{\mathbf k} \newcommand{\bl}{\mathbf l}
\newcommand{\bm}{\mathbf m} \newcommand{\bn}{\mathbf n}
\newcommand{\bo}{\mathbf o} \newcommand{\bp}{\mathbf p}
\newcommand{\bq}{\mathbf q} \newcommand{\br}{\mathbf r}
\newcommand{\bs}{\mathbf s} \newcommand{\bt}{\mathbf t}
\newcommand{\bu}{\mathbf u} \newcommand{\bv}{\mathbf v}
\newcommand{\bw}{\mathbf w} \newcommand{\bx}{\mathbf x}
\newcommand{\by}{\mathbf y} \newcommand{\bz}{\mathbf z}
\newcommand{\bA}{\mathbf A} \newcommand{\bB}{\mathbf B}
\newcommand{\bC}{\mathbf C} \newcommand{\bD}{\mathbf D}
\newcommand{\bE}{\mathbf E} \newcommand{\bF}{\mathbf F}
\newcommand{\bG}{\mathbf G} \newcommand{\bH}{\mathbf H}
\newcommand{\bI}{\mathbf I} \newcommand{\bJ}{\mathbf J}
\newcommand{\bK}{\mathbf K} \newcommand{\bL}{\mathbf L}
\newcommand{\bM}{\mathbf M} \newcommand{\bN}{\mathbf N}
\newcommand{\bO}{\mathbf O} \newcommand{\bP}{\mathbf P}
\newcommand{\bQ}{\mathbf Q} \newcommand{\bR}{\mathbf R}
\newcommand{\bS}{\mathbf S} \newcommand{\bT}{\mathbf T}
\newcommand{\bU}{\mathbf U} \newcommand{\bV}{\mathbf V}
\newcommand{\bW}{\mathbf W} \newcommand{\bX}{\mathbf X}
\newcommand{\bY}{\mathbf Y} \newcommand{\bZ}{\mathbf Z}

\newcommand{\cA}{\mathcal A} \newcommand{\cB}{\mathcal B}
\newcommand{\cC}{\mathcal C} \newcommand{\cD}{\mathcal D}
\newcommand{\cE}{\mathcal E} \newcommand{\cF}{\mathcal F}
\newcommand{\cG}{\mathcal G} \newcommand{\cH}{\mathcal H}
\newcommand{\cI}{\mathcal I} \newcommand{\cJ}{\mathcal J}
\newcommand{\cK}{\mathcal K} \newcommand{\cL}{\mathcal L}
\newcommand{\cM}{\mathcal M} \newcommand{\cN}{\mathcal N}
\newcommand{\cO}{\mathcal O} \newcommand{\cP}{\mathcal P}
\newcommand{\cQ}{\mathcal Q} \newcommand{\cR}{\mathcal R}
\newcommand{\cS}{\mathcal S} \newcommand{\cT}{\mathcal T}
\newcommand{\cU}{\mathcal U} \newcommand{\cV}{\mathcal V}
\newcommand{\cW}{\mathcal W} \newcommand{\cX}{\mathcal X}
\newcommand{\cY}{\mathcal Y} \newcommand{\cZ}{\mathcal Z}


%%%%%%%%%%%%%%Start%%%%%%%%%%%%%Start%%%%%%%%%%%Start%%%%%%%%%%%%%%%Start%%%%%%%%%%%%%%%%%%%%%%%%%Start%%%%%%%%%%%%%%%%
%%%%%%%%%%%%%%Start%%%%%%%%%%%%%Start%%%%%%%%%%%Start%%%%%%%%%%%%%%%Start%%%%%%%%%%%%%%%%%%%%%%%%%Start%%%%%%%%%%%%%%%%
%%%%%%%%%%%%%%Start%%%%%%%%%%%%%Start%%%%%%%%%%%Start%%%%%%%%%%%%%%%Start%%%%%%%%%%%%%%%%%%%%%%%%%Start%%%%%%%%%%%%%%%%
%\documentclass[12pt,oneside]{article}

\usepackage{pdfpages}
%--------------
\usepackage{enumitem}
%-------------Tasks
%\usepackage{tasks} %\begin{tasks} \item \end{tasks}
%\bfseries Horizontal list: a = alphabetical \normalfont
%\begin{tasks}[counter-format = {tsk[a].},label-offset = {0.6em},label-format = {\bfseries}](6)
%\task One
%\task Two
%\task Three
%\task Four
%\task Five
%\task Six
%\task Seven
%\task Eight
%\task Nine
%\task Ten
%\end{tasks}
%\vglue5mm
%\bfseries Horizontal list: A = Alphabetical \normalfont
%\begin{tasks}[counter-format = {(tsk[A])},label-offset = {0.8em},label-format = {\bfseries}](3)
%\task One
%\task Two
%\task Three
%\task Four
%\task Five
%\task Six
%\task Seven
%\task Eight
%\task Nine
%\task Ten
%\end{tasks}



%___________________________
\usepackage[margin=2.5cm]{geometry}

\geometry{hmargin=3cm,vmargin=2cm}
\usepackage{tikz}
\def\width{18}
\def\hauteur{13}


\pagestyle{plain}

%%%%%%%%%%%%%%Start%%%%%%%%%%%%%Start%%%%%%%%%%%Start%%%%%%%%%%%%%%%Start%%%%%%%%%%%%%%%%%%%%%%%%%Start%%%%%%%%%%%%%%%%
%%%%%%%%%%%%%%Start%%%%%%%%%%%%%Start%%%%%%%%%%%Start%%%%%%%%%%%%%%%Start%%%%%%%%%%%%%%%%%%%%%%%%%Start%%%%%%%%%%%%%%%%
%%%%%%%%%%%%%%Start%%%%%%%%%%%%%Start%%%%%%%%%%%Start%%%%%%%%%%%%%%%Start%%%%%%%%%%%%%%%%%%%%%%%%%Start%%%%%%%%%%%%%%%%

\usepackage{fancyhdr}

\pagestyle{fancy}
\fancyhf{}
\rhead{}
\chead{\includegraphics[scale=.1]{snhu_logo.png}}
\begin{document}

\title{\sf MAT 230 Exam Two}%



%\thm{bbjh}


\begin{center}
\includegraphics[scale=.1]{snhu_logo.png}
\end{center}

%\thm{bbjh}
\maketitle
This document is proprietary to Southern New Hampshire University. It and the problems within may not be posted on any non-SNHU website.\\\\\\\\
\begin{center}\doublespacing
%Enter your name below this line:
7-3 Exam Two\\
Nicholas Shaner\\
SNHU\\
MAT-230-15848-M01 Discrete Mathematics\\
Prof. Kirsten Messer\\
December 15, 2024\\
\end{center}

\begin{center}
\rule{\textwidth}{0.4pt}
\end{center}
\newpage
\section*{}
\section*{}
Directions: Type your solutions into this document and be sure to show all steps for arriving at your solution. Just giving a final number may not receive full credit.
\\
\section*{Problem 1}
 \noindent
 This question has 2 parts.
 \subsection*{Part 1:}
 Suppose that $F$ and $X$ are events from a common sample space with $P(F) \neq 0$ and $P(X) \neq 0$.
 \\
 \begin{enumerate}[label=(\alph*)]
     \item Prove that $P(X) = P(X|F)P(F) + P(X|\bar{F})P(\bar{F})$. Hint: Explain why $P(X|F)P(F) = P(X \cap F)$ is another way of writing the definition of conditional probability, and then use that with the logic from the proof of Theorem 4.1.1.
     \\\\
     %Enter your answer below this comment line.
     Using the definition of conditional probability:\\
     $$p(X|F) = \frac{p(X \cap F)}{p(F)}$$
     Multiply both sides by $p(F)$:\\
     $$p(X|F)p(F) = p(X \cap F)$$\\
     The same definition appplies to the conditional probability of $\overline{F}$:\\
     $$p(X|\overline{F}) = \frac{X \cap \overline{F}}{p(\overline{F})}$$
     $$p(X|\overline{F})p(\overline{F}) = p(X \cap \overline{F})$$\\
     Since the law of total probability states:\\
     $$p(X) = p(X \cap F) + p(X \cap \overline{F})$$\\
     Substituting the conditional probabilities proves:\\
     $$p(X) = p(X \cap F) + p(X \cap \overline{F})\;=\;p(X|F)p(F)\,+\,p(X|\overline{F})p(\overline{F})$$
     \\\\
     \item Explain why $P(F|X) = P(X|F)P(F)/P(X)$ is another way of stating Theorem 4.2.1 Bayes’ Theorem.
     \\\\
     %Enter your answer below this comment line.
     Bayes' Theorem states:\\
     $$p(F|X)=\frac{p(X|F)p(F)}{p(X|F)p(F)+p(X|\overline{F})p(\overline{F})}$$\\
     Using the definitions of probability previously evaluated:\\
     $$p(X|F)p(F) = p(X \cap F)$$
     And:\\\\
     $$p(X|\overline{F})p(\overline{F}) = p(X \cap \overline{F})$$\\\\
     \vspace{0.2in}
     Therefore, Bayes' Theorem equivalent to:\\
     $$p(F|X)=\frac{p(X \cap F)}{p(X \cap F)+p(X \cap \overline{F})}$$\\
     \\\\
     As $F$ and $\overline{F}$ are mutually exclusive and $F+\overline{F}=1$, the denominator can be expressed:\\
     $$p(X \cap F)+p(X \cap \overline{F})=p((X \cap F)\cup (X \cap \overline{F}))=p(X)$$\\
     Therefore, substituting the equivalencies found back into the original equation:\\
     $$p(F|X) = \frac{p(X \cap F)}{p(X)} \equiv \frac{p(F|X)p(F)}{p(X)}$$
     \\\\
 \end{enumerate}
 \subsection*{Part 2:}
 A website reports that 70\% of its users are from outside a certain country. Out of their users from outside the country, 60\% of them log on every day. Out of their users from inside the country, 80\% of them log on every day.
 \\
 \begin{enumerate}[label=(\alph*)]
 \item What percent of all users log on every day? Hint: Use the equation from Part 1 (a).
 \\\\
 %Enter your answer below this comment line.
 Using the equation from Part 1 (a):\\
 $$p(X)\,=\,p(X|F)p(F)\,+\,p(X|\overline{F})p(\overline{F})$$\\
 Substituting the given information:\\
 $$p(X)\,=\,(0.6 * 0.7) + (0.8 * 0.3)\,=\,0.42\,+\,0.24\,=\,0.66$$\\
 \begin{center}Therefore:\end{center}
 \vspace{0.1in}
 $$p(X)\,=\,0.66\,=\,66\%$$\\
 {\Large \bf 66\%} of all users log on every day.
 \\\\
 \item Using Bayes’ Theorem, out of users who log on every day, what is the probability that they are from inside the country?
 \\\\
 %Enter your answer below this comment line.
 Substituting the known values evaluated in the previous questions:\\
 $$p(F|X)=\frac{p(X|F)p(F)}{p(X|F)p(F)+p(X|\overline{F})p(\overline{F})}$$
 $$p(F|X)=\left ( \frac{0.24}{0.66}\right )$$
 $$p(F|X)=0.\overline{36}$$
 Therefore, there is a {\Large \bf 36\%} probability of the users who log on every day being from inside the country.
 \end{enumerate}
\newpage

~\\
  \section*{Problem 2}
 \noindent
 This question has 2 parts.
 \subsection*{Part 1:}
 The drawing below shows a Hasse diagram for a partial order on the set:
 \\
   $\{A, \;B,\; C,\; D,\; E,\; F,\; G,\; H,\; I, \; J\}$
 \begin{center}
 \includegraphics[width=2.5in]{NewHasse}
 \end{center}
 {\color{blue} {\bf Figure 1:} \emph{A Hasse diagram shows 10 vertices and 8 edges. The vertices, represented by dots, are as follows:  vertex J is upward of vertex H; vertex H is upward of vertex I; vertex B is inclined upward to the left of vertex A; vertex C is upward of vertex B; vertex D is inclined upward to the right of vertex C; vertex E is inclined upward to the left of vertex F; vertex G is inclined upward to the right of vertex E. The edges, represented by line segments between the vertices are as follows: 3 vertical edges connect the following vertices: B and C, H and I, and H and J; 5 inclined edges connect the following vertices: A and B, C and D, D and E, E and F, and E and G. 
  }
  }
  \\\\
 Determine the properties of the Hasse diagram based on the following questions:

  \begin{enumerate}[label=(\alph*)]
    \item What are the minimal elements of the partial order?
\\\\
  %Enter your answer below this comment line.
  The minimal elements of the given diagram are: I, A, F
\\\\
    \item What are the maximal elements of the partial order?
\\\\
  %Enter your answer below this comment line.
  The maximal elements of the given diagram are: J, D, G
\\\\
    \item Which of the following pairs are comparable?
\[(A,\, D),\; (J,\, F),\; (B,\, E),\; (G,\, F),\; (D,\, B),\; (C,\, F),\; (H,\, I), (C,\, E)\]
\\\\
  %Enter your answer below this comment line.
  The following pairs from the given list are comparable: (A, D), (G, F), (D, B), (H, I)
\\\\
   \end{enumerate}
   \newpage
~\\
\subsection*{Part 2:}
Consider the partial order with domain $\{3,\, 5,\, 6, \,7,\, 10,\, 14,\, 20,\, 30,\, 60,\, 70\}$ and with $x\,\leq \,y$ if $x$ evenly divides $y$. Select the correct Hasse diagram for the partial order.\\

\begin{enumerate}[label=(\alph*)]
\item
\fbox{
\includegraphics[height=3in]{Figure2}\\

}
\\\\
{\color{blue}{\bf Figure 2:} \emph{A Hasse diagram shows a set of elements {3; 5; 6; 7; 10; 14; 20; 30; 60, 70}. There are lines connecting 3 and 6, 6 and 30, 30 and 60, 5 and 10, 10 and 20, 20 and 60, 10 and 70, 7 and 14, 14 and 70.
}
}
\\
\\
  %Enter your answer below this comment line.
  This is {\large \bf NOT} the correct Hasse diagram as there should be an edge between $30$ and $10$.
\\\\
\newpage
~\\~\\
\item
\fbox{
 \includegraphics[height=3in]{Figure3}
}
\\\\
{\color{blue}{\bf Figure 3:} \emph{A Hasse diagram shows a set of elements {3; 5; 6; 7; 10; 14; 20; 30; 60, 70}. There are lines connecting 3 and 6, 6 and 30, 30 and 60, 5 and 10, 10 and 30, 10 and 20, 20 and 60, 10 and 70, 7 and 14, 14 and 70.
}
}
\\
\\
  %Enter your answer below this comment line.
  This {\large \bf IS} the correct Hasse diagram as it relates all elements in the set in which $x$ evenly divides $y$.
\\\\
\newpage
~\\~\\
\item
\fbox{
 \includegraphics[height=3in]{Figure4}\\
}
\\\\
{\color{blue}{\bf Figure 4:} \emph{A Hasse diagram shows a set of elements {3; 5; 6; 7; 10; 14; 20; 30; 60, 70}. There are lines connecting 3 and 6, 6 and 30, 30 and 60, 5 and 10, 10 and 30, 10 and 20, 20 and 60, 20 and 70, 7 and 14, 14 and 70.
}
}
\\
\\
  %Enter your answer below this comment line.
  This is {\large\bf NOT} the correct Hasse diagram as $20$ does not evenly divide $70$.
\\\\
\newpage
~\\~\\
\item
\fbox{
\includegraphics[height=3in]{Figure5}
}
\\\\
{\color{blue}{\bf Figure 5:} \emph{A Hasse diagram shows a set of elements {3; 5; 6; 7; 10; 14; 20; 30; 60, 70}. There are lines connecting 3 and 6, 6 and 30, 30 and 60, 5 and 10, 10 and 30, 10 and 20, 20 and 30, 20 and 60, 10 and 70, 7 and 14, 14 and 70.
}
}
\\\\
  %Enter your answer below this comment line.
  This is {\large\bf NOT} the correct Hasse diagram as $20$ does not evenly divide $30$.
\\\\

\end{enumerate}
  \newpage
~\\
  \section*{Problem 3}
  A car dealership sells cars that were made in 2015 through 2020. Let the cars for sale be the domain of a relation R where two cars are related if they were made in the same year.

  \begin{enumerate}[label=(\alph*)]
    \item Prove that this relation is an equivalence relation.
\\\\
  %Enter your answer below this comment line.
  To prove an equivalence relation, reflexivity, symmetry, and transivity need to be proven.\\
  In the given relationship:\\
  - Since each car built in a certain year can be related to itself, ($aRa$) is confirmed and the relationship is reflexive.\\
  - Since multiple cars are created within the same year, car $x$ is related to car $y$ ($xRy$), and car $y$ is related to car $x$ (yRx), therefore the relationship is symmetric.\\
  - As multiple cars are made each year, car $x$ is related to car $y$ ($xRy$), and car $y$ is related to car $z$ ($yRz$), therefore car $x$ is related to car $z$ ($xRz$), meaning the relationship is transitive.\\
  Since all three conditions are upheald, the relationship is an equivalence relation.
\\\\
    \item Describe the partition defined by the equivalence classes.
\\\\
  %Enter your answer below this comment line.
  The equivalencies classes would be the subsets of all cars made within the same year. The partition created is each of these subsets by manufacturing year 
  and the union of all subsets would be equivalent to the entire domain of all vehicles for sale.
  \end{enumerate}
\newpage
~\\
  \section*{Problem 4}
 Analyze each graph below to determine whether it has an Euler circuit and/or an Euler trail.
 \begin{itemize}
     \item If it has an Euler circuit, specify the nodes for one.
     \item If it does not have an Euler circuit, justify why it does not.
     \item If it has an Euler trail, specify the nodes for one.
     \item If it does not have an Euler trail, justify why it does not.
 \end{itemize}
  \begin{enumerate}[label=(\alph*)]
\item 
\fbox{
 \includegraphics[width=4in]{Figure6}
}
\\\\
{\color{blue} {\bf Figure 6:} \emph{An undirected graph has 6 vertices, a through f. There are 8-line segments that are between the following vertices: a and b, a and c, a and d, a and f, b and c, b and e, b and f, d and e. 
  }
}\\\\
%Enter your answer below this comment line.
Examining the undirected graph, it can be determined to be an Euler circuit. Each vertice is of degree two, and all edges and vertices can be traced by a 
closed walk starting and ending at the same vertex without repeating any edges.\\
An example walk of an Euler circuit within the graph starting at vertex (a) is:\\\\
$\langle\,a,\, b,\, c,\, a,\, d,\, e,\, b,\, f,\, a\, \rangle$
\\\\
   \newpage
~\\~\\
\item
\fbox{
\includegraphics[width=4in]{Figure7}
}
\\\\
{\color{blue} {\bf Figure 7:} \emph{
An undirected graph has 6 vertices, a through f. There are 9-line segments that are between the following vertices: a and b, a and c, a and d, a and f, b and e, b and f, c and d, d and e, d and f. }
}
\\\\
%Enter your answer below this comment line.
Observing the given graph, is can be seen that exactly 2 vertices have a degree of 3 which is a typical characteristic of an Euler trail in which the same restriction as a circuit are applied but the walk starts and ends at different vertices.\\
An example of a Euler trail within he graph is:\\\\
$\langle \,b,\,a,\,d,\,e,\,b,\,f,\,a,\,c,\,d,\,f\, \rangle$
\\\\
   \newpage
~\\~\\
\item 
\fbox{
\includegraphics[width=4in]{Figure8}
}
\\\\
{\color{blue} {\bf Figure 8: } \emph{An undirected graph has 5 vertices, a through e. There are 4-line segments that are between the following vertices: b and c, b and e, c and d, d and e. 
  }
}
\\\\
%Enter your answer below this comment line.
Examining the graph, it is immediately obvious to not have an Euler circuit, or an Euler trail as the graph is {\bf NOT} connected and there are no edges connecting the components.
\\\\

\newpage
~\\~\\
\item 
\fbox{
 \includegraphics[width=4in]{Figure9}
}
\\\\
{\color{blue} {\bf Figure 9:} \emph{An undirected graph has 7 vertices, a through g. There are 10-line segments that are between the following vertices: a and b, a and c, a and f, b and c, b and f, c and d, c and g, d and e, d and f, f and g. 
  }
}
\\\\
%Enter your answer below this comment line.
Looking at the suppled graph, again it can be quickly determined to not be a Euler circuit, or Euler trail as the graph has four vertices of odd number degrees (a, b, d, e).\\
Due to this, the graph does not meet the requirement of an Euler circuit where only even degree vertices exist, nor a Euler trail where exactly two vertices exist with odd degrees. 
\\\\


  \end{enumerate}
\newpage
~\\
  \section*{Problem 5}
  Use Prim's algorithm to compute the minimum spanning tree for the weighted graph. Start the algorithm at vertex A. Explain and justify each step as you add an edge to the tree.
\\
\includegraphics[width=5in]{prim}
\\\\
{\color{blue} {\bf Figure 10:} \emph{A weighted graph shows 5 vertices, represented by circles, and 6 edges, represented by line segments. Vertices A, B, C, and D are placed at the corners of a rectangle, whereas vertex E is at the center of the rectangle. The edges, A B, B D, A C, C D, A E, and E C, have the weights, 7, 3, 2, 4, 5, and 6, respectively.
  }
}
\\\\
%Enter your answer below this comment line.
Starting at vertex $A$, the edges incident to the vertex are $(A, B)$, $(A, E)$ and $(A, C)$. With a weighted value of $2$, $(A, C)$ is added.\\
From vertex $C$, the eligible edges incident to the vertex are $(C, D)$ and $(C, E)$. With a lower weighted value of $4$, $(C, D)$ is added.\\
The next lowest weighted eligible edge added is $(D, B)$ with a weight of $3$.\\
Lastly, to incorporate vertex $E$, the lowest weight eligible edge added is $(A, E)$.\\
Therefore, the final list of edges to create the minimum spanning tree is: $\{A, C\}, \{C, D\}, \{D, B\}, \{A, E\}$\\\\
The final computed value of the minimum spanning tree is: {\Large \bf 14}.
\\\\
 \newpage
~\\
  \section*{Problem 6}
A lake initially contains 1000 fish. Suppose that in the absence of predators or other causes of removal, the fish population increases by 10\% each month. However, factoring in all causes, 80 fish are lost each month.\\

Give a recurrence relation for the population of fish after $n$ months. How many fish are there after 5 months? If your fish model predicts a non-integer number of fish, round down to the next lower integer.
\\\\
%Enter your answer below this comment line.
The recurrence relation for the population of fish after $n$ months is:\\
$${\bf F(n) = (1.1*F(n-1))-80}$$\\
This calculates 110\% of the previous months total fish population then minus 80 fish for the factored losses per month.\\\\
To calculate for fish population after 5 months:\\\\
\begin{equation}\nonumber
\begin{aligned}
F(0)&=1000\\
F(1)&=(1.1 * 1000)-80 = 1020\\
F(2)&=(1.1 * 1020)-80 = 1042\\
F(3)&=(1.1 * 1042)-80 = 1062.2\\
F(4)&=(1.1 * 1062.2)-80 = 1084.42\\
F(5)&=(1.1*1084.42)-80 = 1102.862\\
\end{aligned}
\end{equation}
\\\\
Rounding down to the nearest whole integer, the fish population after 5 months is: {\Large \bf 1102} fish.
\\\\
\end{document}

