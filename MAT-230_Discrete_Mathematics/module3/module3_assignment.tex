% ----------------------------------------------------------------
% AMS-LaTeX Paper ************************************************
% **** -----------------------------------------------------------
%\documentclass{amsart}
%\usepackage{txfonts}
%\documentclass[12pt,oneside]{article}
\documentclass{amsart}
\usepackage{graphicx}
\usepackage{enumitem}
\usepackage{setspace}
% ----------------------------------------------------------------
\vfuzz2pt % Don't report over-full v-boxes if over-edge is small
\hfuzz2pt % Don't report over-full h-boxes if over-edge is small
% THEOREMS -------------------------------------------------------
\newtheorem{thm}{Theorem}[section]
\newtheorem{cor}[thm]{Corollary}
\newtheorem{lem}[thm]{Lemma}
\newtheorem{prop}[thm]{Proposition}
\theoremstyle{definition}
\newtheorem{defn}[thm]{Definition}
\theoremstyle{Exercise}
\newtheorem{ex}[thm]{Exercise}
\theoremstyle{remark}
\newtheorem{rem}[thm]{Remark}
\theoremstyle{rule}
\newtheorem{rul}[thm]{Rule}

\numberwithin{equation}{section}
% MATH -----------------------------------------------------------
\newcommand{\norm}[1]{\left\Vert#1\right\Vert}
\newcommand{\abs}[1]{\left\vert#1\right\vert}
\newcommand{\set}[1]{\left\{#1\right\}}
\newcommand{\Real}{\mathbb R}
\newcommand{\Z}{\mathbb Z}
\newcommand{\To}{\longrightarrow}
\newcommand{\BX}{\bB(X)}
\newcommand{\A}{\mathcal{A}}
% ----------------------------------------------------------------

% define some simple, commonly-used commands
\newcommand{\eps}{\varepsilon}
\newcommand{\dsum}{\displaystyle\sum}
\newcommand{\dint}{\displaystyle\int}

\newcommand{\pdr}[2]{\dfrac{\partial{#1}}{\partial{#2}}}
\newcommand{\pdrr}[2]{\dfrac{\partial^2{#1}}{\partial{#2}^2}}
\newcommand{\pdrt}[3]{\dfrac{\partial^2{#1}}{\partial{#2}{\partial{#3}}}}
\newcommand{\dr}[2]{\dfrac{d{#1}}{d{#2}}}
\newcommand{\aver}[1]{\langle {#1} \rangle}
\newcommand{\Baver}[1]{\Big\langle {#1} \Big\rangle}

\newcommand{\bzero}{\mathbf 0}
\newcommand{\bGamma}{\mbox{\boldmath{$\Gamma$}}}
\newcommand{\btheta}{\boldsymbol \theta}
\newcommand{\bchi}{\mbox{\boldmath{$\chi$}}}
\newcommand{\bnu}{\boldsymbol \nu}
\newcommand{\bmu}{\boldsymbol \mu}
\newcommand{\brho}{\mbox{\boldmath{$\rho$}}}
\newcommand{\bxi}{\boldsymbol \xi}
\newcommand{\bnabla}{\boldsymbol \nabla}
\newcommand{\bOm}{\boldsymbol \Omega}
\newcommand{\blambda}{\boldsymbol \lambda}
\newcommand{\bsigma}{\boldsymbol \sigma}

\newcommand{\bbR}{\mathbb R}
\newcommand{\bbC}{\mathbb C}
\newcommand{\bbQ}{\mathbb Q}
\newcommand{\bbN}{\mathbb N}
\newcommand{\bbZ}{\mathbb Z}

\newcommand{\ba}{\mathbf a} \newcommand{\bb}{\mathbf b}
\newcommand{\bc}{\mathbf c} \newcommand{\bd}{\mathbf d}
\newcommand{\be}{\mathbf e} \newcommand{\bff}{\mathbf f}
\newcommand{\bg}{\mathbf g} \newcommand{\bh}{\mathbf h}
\newcommand{\bi}{\mathbf i} \newcommand{\bj}{\mathbf j}
\newcommand{\bk}{\mathbf k} \newcommand{\bl}{\mathbf l}
\newcommand{\bm}{\mathbf m} \newcommand{\bn}{\mathbf n}
\newcommand{\bo}{\mathbf o} \newcommand{\bp}{\mathbf p}
\newcommand{\bq}{\mathbf q} \newcommand{\br}{\mathbf r}
\newcommand{\bs}{\mathbf s} \newcommand{\bt}{\mathbf t}
\newcommand{\bu}{\mathbf u} \newcommand{\bv}{\mathbf v}
\newcommand{\bw}{\mathbf w} \newcommand{\bx}{\mathbf x}
\newcommand{\by}{\mathbf y} \newcommand{\bz}{\mathbf z}
\newcommand{\bA}{\mathbf A} \newcommand{\bB}{\mathbf B}
\newcommand{\bC}{\mathbf C} \newcommand{\bD}{\mathbf D}
\newcommand{\bE}{\mathbf E} \newcommand{\bF}{\mathbf F}
\newcommand{\bG}{\mathbf G} \newcommand{\bH}{\mathbf H}
\newcommand{\bI}{\mathbf I} \newcommand{\bJ}{\mathbf J}
\newcommand{\bK}{\mathbf K} \newcommand{\bL}{\mathbf L}
\newcommand{\bM}{\mathbf M} \newcommand{\bN}{\mathbf N}
\newcommand{\bO}{\mathbf O} \newcommand{\bP}{\mathbf P}
\newcommand{\bQ}{\mathbf Q} \newcommand{\bR}{\mathbf R}
\newcommand{\bS}{\mathbf S} \newcommand{\bT}{\mathbf T}
\newcommand{\bU}{\mathbf U} \newcommand{\bV}{\mathbf V}
\newcommand{\bW}{\mathbf W} \newcommand{\bX}{\mathbf X}
\newcommand{\bY}{\mathbf Y} \newcommand{\bZ}{\mathbf Z}

\newcommand{\cA}{\mathcal A} \newcommand{\cB}{\mathcal B}
\newcommand{\cC}{\mathcal C} \newcommand{\cD}{\mathcal D}
\newcommand{\cE}{\mathcal E} \newcommand{\cF}{\mathcal F}
\newcommand{\cG}{\mathcal G} \newcommand{\cH}{\mathcal H}
\newcommand{\cI}{\mathcal I} \newcommand{\cJ}{\mathcal J}
\newcommand{\cK}{\mathcal K} \newcommand{\cL}{\mathcal L}
\newcommand{\cM}{\mathcal M} \newcommand{\cN}{\mathcal N}
\newcommand{\cO}{\mathcal O} \newcommand{\cP}{\mathcal P}
\newcommand{\cQ}{\mathcal Q} \newcommand{\cR}{\mathcal R}
\newcommand{\cS}{\mathcal S} \newcommand{\cT}{\mathcal T}
\newcommand{\cU}{\mathcal U} \newcommand{\cV}{\mathcal V}
\newcommand{\cW}{\mathcal W} \newcommand{\cX}{\mathcal X}
\newcommand{\cY}{\mathcal Y} \newcommand{\cZ}{\mathcal Z}


%%%%%%%%%%%%%%Start%%%%%%%%%%%%%Start%%%%%%%%%%%Start%%%%%%%%%%%%%%%Start%%%%%%%%%%%%%%%%%%%%%%%%%Start%%%%%%%%%%%%%%%%
%%%%%%%%%%%%%%Start%%%%%%%%%%%%%Start%%%%%%%%%%%Start%%%%%%%%%%%%%%%Start%%%%%%%%%%%%%%%%%%%%%%%%%Start%%%%%%%%%%%%%%%%
%%%%%%%%%%%%%%Start%%%%%%%%%%%%%Start%%%%%%%%%%%Start%%%%%%%%%%%%%%%Start%%%%%%%%%%%%%%%%%%%%%%%%%Start%%%%%%%%%%%%%%%%

\usepackage{fancyhdr}

\pagestyle{fancy}
\fancyhf{}
\rhead{}
\chead{\includegraphics[scale=.1]{snhu_logo.png}}

\begin{document}
\begin{center}
\includegraphics[scale=.1]{snhu_logo.png}
\end{center}
\title{\sf Module Three Problem Set}%


%\thm{bbjh}
\maketitle
This document is proprietary to Southern New Hampshire University. It and the problems within may not be posted on any non-SNHU website.\\\\\\\\
\begin{center}\doublespacing
%Enter your name below this line:
3-4 Problem Set\\
Nicholas Shaner\\
SNHU\\
MAT-230-15848-M01 Discrete Mathematics\\
Prof. Kirsten Messer
November 17, 2024\\
\end{center}


\begin{center}
\rule{\textwidth}{0.4pt}
\end{center}


\newpage
\section*{}
\section*{}
Directions: Type your solutions into this document and be sure to show all steps for arriving at your solution. Just giving a final number may not receive full credit.
\\\\



%--------------------------------------------------------------------------------------------------

\section*{Problem 1}

A 125-page document is being printed by five printers. Each page will be printed exactly once.
 \begin{enumerate}[label=(\alph*)]
 \item  Suppose that there are no restrictions on how many pages a printer can print. How many ways are there for the 125 pages to be assigned to the five printers?\\\\
{\it One possible combination: printer A prints out pages 2-50, printer B prints out pages 1 and 51-60, printer C prints out 61-80 and 86-90, printer D prints 
out pages 81-85 and 91-100, and printer E prints out pages 101-125.}\\\\
%Enter your answer below this comment line.
  - With no restriction as to which printer each page is assigned, or how many pages each printer can, or should be assigned:\\
  - Let the total number of pages be represented by variable $r$.\\
  - Let the available number of printers be represented by variable $n$.\\
  - As each single page has 5 available printers to be dispatched to, the number of ways $r$ pages can be assigned to the $n$ printers is 
  ($5*5*5*5\dots$) for each page, equivalent to $n^r$.\\
  - Therefore, there are $5^{125}$ ways to assign the 125 pages among 5 prints.
        \\\\
 \item Suppose the first and the last page of the document must be printed in color, and only two printers are able to print in color. The two color printers 
 can also print black and white. How many ways are there for the 125 pages to be assigned to the five printers?\\\\
%Enter your answer below this comment line.
  - With 2 pages restricted to 2 printers, the total possible ways of distributing these restricted pages is $2^2$ or $4$\\
  - The remaining 123 pages then have no direct restrictions as to which printers they should be assigned to. With reference to \textit{Problem 1.(a)} above,
  this would indicate that there are $5^{123}$ of assigning these remaining pages.\\
  - Therefore, the total number of possible page assignments would be $4*5^{123}$.\\
        \\\\
 \item Suppose that all the pages are black and white, but each group of 25 consecutive pages (1-25, 26-50, 51-75, 76-100, 101-125) must be assigned to the same 
 printer. Each printer can be assigned 0, 25, 50, 75, 100, or 125 pages\\ to print.
How many ways are there for the 125 pages to be assigned to the five printers?\\\\
%Enter your answer below this comment line.
\vspace*{0.5in}\\
  - As the 125 pages are now broken down into 5 subsets, the subsets can be viewed as individual elements.
  - There are again no restrictions to which printer the subsets are to be assigned, as well as how many subsets can be assigned to each printer.\\
  - Therefore each subset has 5 possible assignment options, signified by the equation $5^5$ or 3125 different ways to distribute the pages.\\ 
        \\\\
   \end{enumerate}
 \newpage
%--------------------------------------------------------------------------------------------------
\vspace*{0.2in}
\section*{Problem 2}
Ten kids line up for recess. The names of the kids are:\\
\begin{center}
 \{Alex, Bobby, Cathy, Dave, Emy, Frank, George, Homa, Ian, Jim\}.\\
\end{center}
Let $S$ be the set of all possible ways to line up the kids. For example, one order might be:
\begin{center}
  (Frank, George, Homa, Jim, Alex, Dave, Cathy, Emy, Ian, Bobby)\\
\end{center}

The names are listed in order from left to right, so Frank is at the front of the line and Bobby is at the end of the line.\\

Let $T$ be the set of all possible ways to line up the kids in which George is ahead of Dave in the line. Note that George does not have to be immediately 
ahead of Dave. For example, the ordering shown above is an element in $T$.\\

Now define a function $f$ whose domain is $S$ and whose target is $T$. Let $x$ be an element of $S$, so $x$ is one possible way to order the kids. If George 
is ahead of Dave in the ordering $x$, then $f(x) = x$. If Dave is ahead of George in $x$, then $f(x)$ is the ordering that is the same as $x$, except that 
Dave and George have swapped places.\\
\begin{enumerate}[label=(\alph*)]
  \item What is the output of $f$ on the following input?\\
  (Frank, George, Homa, Jim, Alex, Dave, Cathy, Emy, Ian, Bobby)\\\\
%Enter your answer below this comment line.
  - As stated in the conditions, when George is ahead of Dave, the equation is $f(x)=x$.\\ 
  - Therefore, the output of\\ $f(Frank, George, Homa, Jim, Alex, Dave, Cathy, Emy, Ian, Bobby)$\\ is:\\ (Frank, George, Homa, Jim, Alex, Dave, Cathy, Emy, Ian, Bobby).\\
\\\\
  \item What is the output of $f$ on the following input?\\
(Emy, Ian, Dave, Homa, Jim, Alex, Bobby, Frank, George, Cathy)\\\\
%Enter your answer below this comment line.
  - It is stated that the equation $f(x)=x$ also respresents the same line up, except George and Dave are in reversed positions.\\
  - Therefore, the output of:\\ $f(Emy, Ian, Dave, Homa, Jim, Alex, Bobby, Frank, George, Cathy)$\\ is $x$, which is equivalent of:\\ $(Frank, George, Homa, Jim, Alex, Dave, Cathy, Emy, Ian, Bobby)$\\
\\\\\
\vspace*{0.3in}
  \item Is the function $f$ a $k$-to-1 correspondence for some positive integer $k$? If so, for what value of $k$? Justify your answer.\\\\
%Enter your answer below this comment line.
  - Directly, \text{\bf{Yes}}, function $f$ is a $k$-to-1 correspondence.\\
  - In evaluation, with no restrictions, the total number of potential line ups is the domain $S$, represented by permutation equation $|S|=10!$.\\
  - As $T$ represents the set of all line ups in which George comes before Dave. $|\overline{T}|$ represents the remaining complement sets, where as $|T|+|\overline{T}|=|S|$\\
  - Let variables $G$ = George, and $D$=Dave. The only possibilities of orientation of $G$ and $D$ in domain $S$ is either $GD$ or $DG$ as $G$ and $D$ must both be present and can not be listed in the same spot in the list at the same time.\\
  - Therefore, the $k$-to-1 correspondence of $f$ of set $T$ over domain $S$ can be evaluated as $|T|=|S|/2$, where $k=2$.
\\\\
  \item There are 3628800 ways to line up the 10 kids with no restrictions on who comes before whom. That is, $|S| =3628800$. Use this fact and the answer to the previous question to determine $|T|$.\\\\
%Enter your answer below this comment line.
- Using the equation found in the previous questions, $|T|=|S|/2$, and the evaluations of $|S| = 3628800$ it can be evaluated that $|T| = 18814400$ for either function of $T$ or $\overline{T}$.
\\\\
\end{enumerate}

   
   \newpage
%--------------------------------------------------------------------------------------------------
\vspace*{0.2in}
\section*{Problem 3}
   
   
Consider the following definitions for sets of characters:
\begin{itemize}
  \item Digits $\;=\; \{ 0,\, 1,\, 2,\, 3,\, 4,\, 5,\, 6,\, 7,\, 8,\, 9 \}$\\
  \item Letters$\; = \;\{ a,\, b,\, c, \,d,\, e,\, f,\, g,\, h,\, i,\, j,\, k,\, l,\, m,\, n,\, o,\, p,\, q,\, r,\, s,\, t,\, u,\, v,\, w,\, x,\, y,\, z \}$\\
  \item Special characters $\;=\; \{ *,\, \&,\, \$,\, \# \}$\\
\end{itemize}

Compute the number of passwords that satisfy the given constraints.
    \begin{enumerate}[label=(\roman*)]
    \item Strings of length 7. Characters can be special characters, digits, or letters, with no repeated characters.\\\\
%Enter your answer below this comment line.
  - $P(40, 7)=\frac{40!}{(40-7)!} = 93,963,542,400$\\
\\\\
    \item Strings of length 6. Characters can be special characters, digits, or letters, with no repeated characters. The first character can not be a special character.\\\\
%Enter your answer below this comment line.
  - As the first character can only be one of 36 potential character from the domain, the potential combinations of the remaining 5 characters of the password with
  repeated characters can be evaluated using the $k-to-1$ rule evaluate $P(39,5)$.\\
  - Therefore, $\frac{39!}{(39-5)!}*36$, which evaluates to 2,487,270,240 potential combinations.
    \end{enumerate}
 \newpage
%--------------------------------------------------------------------------------------------------
\vspace*{0.1in}
\section*{Problem 4}
A group of four friends goes to a restaurant for dinner. The restaurant offers 12 different main dishes.\\
    \begin{enumerate}[label=(\roman*)]
    \item Suppose that the group collectively orders four different dishes to share. The waiter just needs to place all four dishes in the center of the table. How many different possible orders are there for the group?\\\\
%Enter your answer below this comment line.
  - Using a 12 choose 4 subset count, $\frac{12!}{4!(12-4)!}=495$.\\
  \\\\
    \item Suppose that each individual orders a main course. The waiter must remember who ordered which dish as part of the order. It's possible for more than one person to order the same dish. How many different possible orders are there for the group?\\\\
%Enter your answer below this comment line.
  - As there are no restrictions on repeat orders, each individual has the option of the full domain of 12 different dishes.\\
  - Therefore, $12^4 = 20736$ potential combinations of dishes.
\\\\

    \end{enumerate}


How many different passwords are there that contain only digits and lower-case letters and satisfy the given restrictions?\\
      \begin{enumerate}[label=(\roman*), start=3]
    \item Length is 7 and the password must contain at least one digit.\\\\
%Enter your answer below this comment line.
  - As the password only requires at least 1 digit, the remaining 6 characters are unrestricted.\\
  - (0-9) = 10 characters\\
  - (a-z) = 26 characters\\
  - The remaining 6 characters can be evaluated as $36^6$.\\
  - Therefore, $36^6*10 = 21,767,823,360$ possible combinations.
\\\\
     \item Length is 7 and the password must contain at least one digit and at least one letter.\\\\
%Enter your answer below this comment line.
  - Using the same theory as the previous question, 1 character is required to be a digit (0-9), and 1 character a letter (a-z).\\
  - (0-9) = 10 characters\\
  - (a-z) = 26 characters\\
  - The remaining 5 characters can be evaluated as $36^5$\\
  - Therefore, $36^5*10*26 = 15,721,205,760$ possible combinations.\\
    \end{enumerate}
 
 \newpage
%--------------------------------------------------------------------------------------------------
\vspace*{0.1in}
\section*{Problem 5}

A university offers a Calculus class, a Sociology class, and a Spanish class. You are given data below about two groups of students.\\\\
     \begin{enumerate}[label=(\roman*)]
     \item Group 1 contains 170 students, all of whom have taken at least one of the three courses listed above. Of these, 61 students have taken Calculus, 78 have taken Sociology, and 72 have taken Spanish. 15 have taken both Calculus and Sociology, 20 have taken both Calculus and Spanish, and 13 have taken both Sociology and Spanish. How many students have taken all three classes?\\\\
%Enter your answer below this comment line.
  - Given the stated information:
  Calculus = A = 61, Sociology = B = 78, Spanish = C = 72\\
  $A\cap{B}$ = 15, $A\cap{C}$ = 20, $B\cap{C}$ = 13\\
  - Using the Principle of Inclusion-Exclusion:\\
  $|A\cup{B}\cup{C}| = |A|+|B|+|C|-|A\cap{C}|-|B\cap{C}|+|A\cap{B}\cap{C}|$, we can solve for $|A\cap{B}\cap{C}|$\\
  - $170=61+78+72-15-20-13 + x$\\
  - Therefore, $|A\cap{B}\cap{C}| = 7$, students took all three courses. 
\\\\\
   
\item You are given the following data about Group 2. 32 students have taken Calculus, 22 have taken Sociology, and 16 have taken Spanish. 10 have taken both Calculus and Sociology, 8 have taken both Calculus and Spanish, and 11 have taken both Sociology and Spanish. 5 students have taken all three courses while 15 students have taken none of the courses. How many students are in Group 2?\\\\
%Enter your answer below this comment line.
  - Given the stated information of Group 2:\\
  Calculus = A = 32, Sociology = B = 22, Spanish = C = 16, No course taken = 15\\
  $A\cap{B}$ = 10, $A\cap{C}$ = 8, $B\cap{C}$ = 11\\
  $A\cup{B}\cup{C}$ = 5\\
  - Using the Principle of Inclusion-Exclusion:\\
  $|A\cup{B}\cup{C}| = |A|+|B|+|C|-|A\cap{C}|-|B\cap{C}|+|A\cap{B}\cap{C}| + 15$\\
  $|A\cup{B}\cup{C}| = 32 + 22 + 16 - 10 - 8 - 11 + 5 + 15$\\
  - Therefore, $|A\cup{B}\cup{C}|$ = 61, total students in Group B.
\\\\\
         \end{enumerate}
 \newpage
%--------------------------------------------------------------------------------------------------
\vspace*{0.1in}
\section*{Problem 6}
A coin is flipped five times. For each of the events described below, express the event as a set in roster notation. Each outcome is written as a string of length 5 from $\{H,\, T\}$, such as $HHHTH$. Assuming the coin is a fair coin, give the probability of each event.\\
\begin{enumerate}[label=(\alph*)]
\item The first and last flips come up heads.\\\\\
%Enter your answer below this comment line.
  - Total number of outcomes of 5 coil flips: $2^5=32$\\
  - Total unmber of outcomes if first and last flip are H: $2^5-2^2=2^3=8$
  \\\\
  \{HHHHH\;\;HHHTH\;\;HHTHH\;\;HTHHH\\
  HTTHH\;\;HHTTH\;\;HTHTH\;\;\;HTTTH\}
  \\\\
  - Probability of favorable outcome: $\frac{8}{32}=\frac{1}{4}$\\
\\\\\
\item There are at least two consecutive flips that come up heads.\\\\\
%Enter your answer below this comment line.
  - Referencing the first problem, there are 32 potential outcomes.\\
  - 16 of the outcomes validate the experiment\\
  - Therefore, $16/32 = 1/2$
\\\\\
\item The first flip comes up tails and there are at least two consecutive flips that come up heads.\\\\\
%Enter your answer below this comment line.
  - Again referencing the first problem, there are 32 potential outcomes.\\
  - Blocking the first flip as a T, there are 4 remaining flips.\\
  - Blocking 2 additional spaces for the favorable outcome of HH, leaves: $2^5-2^1-2^2 = 2^2 = 4$\\
  - Therefore, $4/32 = 1/8$
\\\\\
\end{enumerate}

 \newpage
%--------------------------------------------------------------------------------------------------
\vspace*{0.1in}
\section*{Problem 7}
An editor has a stack of $k$ documents to review.  The order in which the documents are reviewed is random with each ordering being equally likely. Of the $k$ documents to review, two are named ``Relaxation Through Mathematics'' and ``The Joy of Calculus.'' Give an expression for each of the probabilities below as a function of k. Simplify your final expression as much as possible so that your answer does not include any expressions in the form\\
$
\Big(
 \begin{array}{c}
 a\\
 b
    \end{array}
    \Big)
$.
 \begin{enumerate}[label=(\alph*)]
\item What is the probability that ``Relaxation Through Mathematics'' is first to review?\\\\
%Enter your answer below this comment line.
  - Let variable D represent the domain of books with k number of elements. The total number of combinations equivalent to $k!$\\
  - Total number of favorable outcomes is $1*(k-1)!$\\
  - Therefore, the probability of "Relaxation Through Mathematics" being the first book pulled is $\frac{(k-1)!}{k!} = \frac{1}{k}$.
\\\\
\item What is the probability that ``Relaxation Through Mathematics'' and ``The Joy of Calculus'' are next to each other in the stack?\\\\
%Enter your answer below this comment line.
  - Using the information gained in the first problem, the total number of possible combinations is $k!$.\\
  - The probability of both books being next to eachother in the compination is $((k-1)(k-2))!$\\
  - Therefore, the probability of of both books being pulled one after the other is $\frac{((k-1)(k-2))!}{k!} = \frac{2}{k}$
\\\\
\end{enumerate}


\end{document}
